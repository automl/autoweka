The first step in doing any analysis of experiments is to generate the trajectories of the optimisation run. This can be done by invoking the \classname{autoweka.TrajectoryParser} class, with the arguments of the experiment folder and the seed you want to parse. This step is only required if you did not use the \classname{autoweka.tools.ExperimentRunner} to execute your experiment.

For each seed that the experiment was run with, there should be a file with the naming scheme of \texttt{<ExperimentName>.trajectories.<Seed>}. If you defined any \texttt{trajectoryPointExtras} in your experiment definition, you'll want to run the main method of \texttt{autoweka.TrajectoryPointExtraRunner} with the arguments \\\texttt{<ExperimentFolder>/<ExperimentName>.trajectories.<Seed>} before performing any other analysis. This does runs of the classifer/hyperparameters identified in the trajectory on whatever instance strings you've specified, and stores the result of the error metric and timing information into the trajectory file.

Once all the individual trajectory files are complete, each of these files needs to be merged into a single trajectory group for analysis. Run the main method of \texttt{autoweka.TrajectoryMerger}, with a single argument of the experiment's directory. This produces a single file \texttt{<ExperimentName>.trajectories} inside the experiment's folder.

Finally, to get the best hyper-parameters and method that Auto-WEKA has found on the dataset, run the main method of \texttt{autoweka.tools.GetBestFromTrajectoryGroup}, with the single command line argument pointing at the \texttt{.trajectories} file that was produced in the last step. This will print out information on how many trajectories were used to select the best, what Auto-WEKA thinks the performance of the selected method on the training set is, as well as which algorithm was chosen and the command line arguments that should be given to the algorithm.