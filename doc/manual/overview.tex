
Using Auto-WEKA can be broken down into three main steps. First, you have to build your experiment definition, which tells Auto-WEKA what dataset(s) to run on, as well as what kind of hyperparameter search will be done (either through a model based method, or through something like grid search). Once the definition has been written, the experiment needs to be fully instantiated by having Auto-WEKA detect what kind of classifiers can be used given the definition you wrote. At this stage, Auto-WEKA also resolves all path names to absolute paths, so instantiated experiments may have some difficulty when being moved between different computation environments.  

Once an experiment has been produced, it actually has to be executed. Auto-WEKA takes advantage of multiple cores by running the same experiment with different random seeds, the only requirement is that all the experiments have a similar file system (since Auto-WEKA relies on absolute path names). The user has to start a new run of the experiment for each seed/core that they want to take advantage of.

After the experiment has been executed, the analysis phase occurs. When Auto-WEKA uses a model based optimisation method, it produces a trajectory of hyperparameters that were identified by the optimisation method as being the best at a particular point in time. The simplest form of analysis looks at the best hyperparameters that were found across all seeds, and uses the trained model to make predictions on a new dataset. Additional experiments can be performed on these trajectory points, for example to see if all the trajectory points have a similar performance on a new set of data that the optimisation method did not have access to.

The above three steps can all be completed through the use of a GUI (see Section~\ref{sec:gui}), or through the more flexible command line interface (see Sections~\ref{sec:defining}-\ref{sec:analyzing}).